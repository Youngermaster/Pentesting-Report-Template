\hypertarget{basics}{%
\section{Basics}\label{basics}}

This is not just a \emph{proof of concept}, but a basic utilization of
{[}Pandoc{]}{[}2{]}'s behavior when it comes to {[}Markdown{]}{[}1{]}
processing.

\begin{itemize}
\item
  Pandoc, by default, passes any \LaTeX~code snippets it identifies
  within the Markdown source file to the target document, if that target
  document is a \LaTeX~one (this includes f.e. Beamer or PDF output. (It
  does not pass these snippets to any other output formats, but instead
  drops them.)
\item
  Pandoc, by default, also passes any HTML code snippets it identifies
  within the Markdown sources to the target document, should that be
  HTML based (this includes f.e. EPUB or RevealJS output). (It does not
  pass these snippets to any other output formats, but instead drops
  them.)
\item
  This allows for any specific formatting to be achieved in the target
  document format: \textbf{\emph{(1)}} Insert two versions of the
  snippet, one as HTML, one as \LaTeX. \textbf{\emph{(2)}} The HTML one
  will make it to the HTML-based targets, while \LaTeX~is being dropped;
  the \LaTeX~will make it into \LaTeX-based targets, while the HTML is
  being dropped.
\end{itemize}

\hypertarget{simple-list}{%
\subsection{Simple list}\label{simple-list}}

\begin{itemize}
\item
  First item
\item
  Second item
\item
  Third item
\item
  Fourth item
\end{itemize}

\hypertarget{another-subsection}{%
\subsection{Another subsection}\label{another-subsection}}

This is a \href{https://cwe.mitre.org/data/definitions/200.html}{link}

\hypertarget{subsubsection}{%
\subsubsection{Subsubsection}\label{subsubsection}}

Lorem.

It also supports linked references.
\textbf{\protect\hyperlink{tab:fsttable}{Click here}} to jump to a page
with a table.

\hypertarget{how-it-works}{%
\subsection{How it works}\label{how-it-works}}

It works out of the box:
